\chapter{Conclusão}
\label{cap-status-atual}

% Este capítulo deve apresentar as conclusões do trabalho, retomando objetivos e questões de pesquisa
% Organize o texto de forma clara, mostrando os resultados alcançados

\section{Contexto Geral}
\label{contexto-geral}
% Faça um breve resumo do contexto geral do trabalho
% Retome os principais pontos abordados

\section{Estado Final do Trabalho}
\label{status-final}
% Descreva o estado final alcançado pelo trabalho

\subsection{Questão de Pesquisa}
\label{status-pesquisa}
% Retome as questões de pesquisa e apresente como foram respondidas

\begin{enumerate}
\item Questão 1: Descreva como a primeira questão foi respondida
\item Questão 2: Descreva como a segunda questão foi respondida
\end{enumerate}

\subsection{Objetivos}
\label{status-objetivos}

O cumprimento dos \nameref{1-objetivos} deste trabalho, compreendendo o objetivo geral e os objetivos específicos, é apresentado a seguir.

\subsubsection{Objetivo Geral}
\label{status-objetivo-geral}
% Descreva como o objetivo geral foi alcançado
% Relacione com os resultados obtidos

\subsubsection{Objetivos Específicos}
\label{status-objetivo-especificos}
% Retome cada objetivo específico e descreva como foi alcançado

\begin{itemize}
\item Como Objetivos Específicos de viés Teórico, têm-se:
\begin{enumerate}
\item \textbf{OT1:} Descreva o primeiro objetivo teórico;
\item \textbf{OT2:} Descreva o segundo objetivo teórico;
\item \textbf{OT3:} Descreva o terceiro objetivo teórico; e
\item \textbf{OT4:} Descreva o quarto objetivo teórico.
\end{enumerate}
\item Como Objetivo Específico de viés Prático, tem-se:
\begin{itemize}
\item \textbf{OP1:} Descreva o objetivo prático.
\end{itemize}
\item Como Objetivo Específico de viés Documental, tem-se:
\begin{itemize}
\item \textbf{OD1:} Descreva o objetivo documental.
\end{itemize}
\end{itemize}

\begin{table}[h]
\centering
{\renewcommand{\arraystretch}{1.5}
\scriptsize
\caption[Estado Final dos Objetivos Específicos]{Estado Final dos Objetivos Específicos do TCC.}
\label{status-objetivos-especificos}
\begin{tabular}{cccc}
\hline
\textbf{\begin{tabular}[c]{@{}c@{}}OBJETIVO\\ ESPECÍFICO\end{tabular}} & \textbf{VIÉS} & \textbf{ESTADO FINAL} & \textbf{COMPROBATÓRIO} \\ \hline
OT1 & Teórico & Concluído/Em andamento & Seção X.Y \\ \hline
OT2 & Teórico & Concluído/Em andamento & Seção X.Y \\ \hline
OT3 & Teórico & Concluído/Em andamento & Seção X.Y \\ \hline
OT4 & Teórico & Concluído/Em andamento & Seção X.Y \\ \hline
OP1 & Prático & Concluído/Em andamento & Seção X.Y \\ \hline
OD1 & Documental & Concluído/Em andamento & Seção X.Y \\ \hline
\end{tabular}
\legend{Fonte: Autor.}
}
\end{table}

\section{Atividades Realizadas}
\label{status-atividades-realizadas}

As atividades propostas para a primeira etapa do TCC, discorridas já na seção \nameref{met-atividades-tcc1}, podem ser vistas a seguir com seus respectivos \textit{status} final e comprobatórios.

\begin{sidewaystable}[h]
\centering
\ifodd\value{page}
\begin{turn}{0}
\begin{minipage}{\textwidth}
\centering
{\renewcommand{\arraystretch}{1.5}
\scriptsize
\caption[Estado Final das Atividades Primeira Etapa]{Estado Final das Atividades Propostas para a Primeira Etapa.}
\label{status-primeira-etapa}
\begin{tabular}{ccc}
\hline
\textbf{ATIVIDADE / SUBPROCESSO} & \textbf{ESTADO FINAL} & \textbf{COMPROBATÓRIO} \\ \hline
Revisão Bibliográfica & Concluído & Capítulo X \\ \hline
Coleta de Dados & Concluído & Seção Y.Z \\ \hline
Análise dos Resultados & Concluído & Seção Y.Z \\ \hline
\end{tabular}
\legend{Fonte: Autor.}
}
\end{minipage}
\end{turn}
\else
\begin{turn}{180}
\begin{minipage}{\textwidth}
\centering
{\renewcommand{\arraystretch}{1.5}
\scriptsize
\caption[Estado Final das Atividades Primeira Etapa]{Estado Final das Atividades Propostas para a Primeira Etapa.}
\label{status-primeira-etapa}
\begin{tabular}{ccc}
\hline
\textbf{ATIVIDADE / SUBPROCESSO} & \textbf{ESTADO FINAL} & \textbf{COMPROBATÓRIO} \\ \hline
Revisão Bibliográfica & Concluído & Capítulo X \\ \hline
Coleta de Dados & Concluído & Seção Y.Z \\ \hline
Análise dos Resultados & Concluído & Seção Y.Z \\ \hline
\end{tabular}
\legend{Fonte: Autor.}
}
\end{minipage}
\end{turn}
\fi
\end{sidewaystable}

As atividades propostas para a segunda etapa são listadas com seus respectivos estados finais a seguir.

\begin{table}[h]
\centering
{\renewcommand{\arraystretch}{1.5}
\scriptsize
\caption[Estado Final das Atividades Segunda Etapa]{Estado Final das Atividades Previstas para a Segunda Etapa.}
\label{status-segunda-etapa}
\begin{tabular}{ccc}
\hline
\textbf{ATIVIDADE / SUBPROCESSO} & \textbf{ANDAMENTO} & \textbf{COMPROBATÓRIO} \\ \hline
Desenvolvimento & Concluído & Capítulo X \\ \hline
Testes & Concluído & Seção Y.Z \\ \hline
Documentação & Concluído & Seção Y.Z \\ \hline
\end{tabular}
\legend{Fonte: Autor.}
}
\end{table}

\section{Contribuições}
\label{status-contribuicoes}
Este trabalho oferece algumas contribuições para a comunidade acadêmica e profissional que merecem menção, dentre elas cabem destacar que esta pesquisa:

\begin{itemize}
\item Liste aqui as principais contribuições do seu trabalho;
\item Para cada contribuição, explique seu impacto e relevância;
\item Inclua contribuições teóricas e práticas.
\end{itemize}

\section{Futuros Trabalhos}
\label{status-futuros}
Este trabalho apresenta, também, algumas limitações, que podem ser exploradas em trabalhos futuros. Dentre elas, destacam-se:

\begin{itemize}
\item Liste aqui as possibilidades de trabalhos futuros;
\item Para cada sugestão, explique brevemente como poderia ser desenvolvida;
\item Inclua tanto melhorias do trabalho atual quanto novas direções de pesquisa.
\end{itemize}

\section{Considerações Finais}
\label{status-consideracoes-finais}
Com base nos resultados e nas experiências adquiridas durante este trabalho, algumas considerações finais extras merecem ser traçadas:

\begin{itemize}
\item Apresente reflexões sobre o processo de desenvolvimento do trabalho;
\item Discuta lições aprendidas e insights obtidos;
\item Comente sobre possíveis impactos práticos dos resultados.
\end{itemize}

\section{Resumo do Capítulo}
\label{status-resumo}
% Sintetize os principais pontos abordados na conclusão
% Encerre com uma reflexão final sobre a importância do trabalho