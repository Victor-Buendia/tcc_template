\chapter{Introdução}
\label{1-introducao}

% Este capítulo deve apresentar uma visão geral do trabalho
% Comece com uma contextualização ampla e vá afunilando até seu tema específico

\section{Contextualização}
\label{1-contextualizacao}
% Apresente o contexto geral do problema
% Descreva a área de pesquisa e sua relevância atual

% Exemplo de como usar uma citação:
\begin{citacao}
A engenharia de software tem evoluído significativamente nos últimos anos \cite{inmetro2003}.
\end{citacao}

\section{Questionamentos}
\label{1-questao}
% Liste aqui as questões que seu trabalho pretende responder
% Formule perguntas claras e objetivas
% As questões devem estar alinhadas com seus objetivos

\section{Justificativa}
\label{1-justificativa}
% Explique por que seu trabalho é importante
% Apresente argumentos que justifiquem a pesquisa

\begin{itemize}
\item Primeiro argumento importante sobre a relevância do trabalho
\item Segundo argumento importante sobre o impacto esperado
\item Demais pontos que justificam a pesquisa
\end{itemize}

% Exemplo de citação que reforce seus argumentos:
\begin{citacao}
O desenvolvimento contínuo de novas tecnologias demanda constante atualização das práticas de engenharia \cite{Sommerville2007}.
\end{citacao}

\section{Objetivos}
\label{1-objetivos}
% Apresente uma breve introdução aos objetivos do trabalho

\subsection{Objetivo Geral}
\label{1-objetivos-gerais}
% Descreva o objetivo principal do trabalho
% Use verbos no infinitivo (desenvolver, analisar, propor, etc.)

\subsection{Objetivos Específicos}
\label{1-objetivos-especificos}
% Liste os objetivos específicos que contribuirão para alcançar o objetivo geral

\begin{itemize}
\item Objetivos Específicos de viés Teórico:
\begin{enumerate}
\item Primeiro objetivo teórico (ex: analisar, investigar...)
\item Segundo objetivo teórico (ex: comparar, avaliar...)
\item Demais objetivos teóricos
\end{enumerate}
\item Objetivos Específicos de viés Prático:
\begin{itemize}
\item Objetivo prático (ex: implementar, desenvolver...)
\end{itemize}
\item Objetivos Específicos de viés Documental:
\begin{itemize}
\item Objetivo documental (ex: documentar, registrar...)
\end{itemize}
\end{itemize}

\section{Metodologia}
\label{1-metodologia}
% Apresente um resumo da metodologia utilizada
% Indique o tipo de pesquisa e principais métodos
% Detalhe completo será feito no capítulo de metodologia

\section{Organização da Monografia}
\label{1-organizacao}
% Apresente a estrutura do trabalho, descrevendo brevemente cada capítulo

\begin{itemize}
\item Capítulo \ref{cap-referencial-teorico} \nameref{cap-referencial-teorico}: Fundamentação teórica do trabalho
\item Capítulo \ref{cap-suporte-tecnologico} \nameref{cap-suporte-tecnologico}: Descrição das tecnologias utilizadas
% Continue listando os demais capítulos
\end{itemize}
