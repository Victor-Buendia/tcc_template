\chapter{Metodologia}
\label{cap-metodologia}

Lorem ipsum dolor sit amet, consectetur adipiscing elit. Sed scelerisque porta accumsan. Cras sed vestibulum lorem. Curabitur ac tincidunt urna. Praesent non faucibus risus. Praesent efficitur lacus et euismod venenatis. Nulla nec ante felis. Praesent ex libero, mollis sit amet commodo et, finibus id orci. Pellentesque ut mi accumsan sapien hendrerit maximus. Praesent iaculis sit amet erat a imperdiet. Maecenas ac leo pulvinar lacus posuere pharetra. Sed suscipit, arcu ut commodo elementum, ipsum tellus fermentum dolor, nec aliquam elit tellus et felis. Phasellus lacinia tempus tempus. In dictum mi purus, nec lobortis turpis consequat in. Etiam suscipit leo quis augue vulputate auctor.

\section{Classificação da Pesquisa}
\label{mtd-classificacao-pesquisa}

% \begin{figure}[h]
% 	\centering
% 	\caption[Classificação da Pesquisa Científica]{Tipos de classificação de pesquisa científica e as respectivas classificações deste trabalho.}
% 	\includegraphics[keepaspectratio=true,scale=0.17]{figuras/research_types.eps}
% 	\label{figura-tipos-pesquisa}
% 	\legend{Fonte: \cite[p.33-44]{gerhardt_metodos_2009}. Figura elaborada pelo autor.}
% \end{figure}

\subsection{Abordagem de Pesquisa}
\label{mtd-abordagem-pesquisa}


\subsection{Objetivo de Pesquisa}
\label{mtd-objetivo-pesquisa}


\subsection{Natureza de Pesquisa}
\label{mtd-natureza-pesquisa}

\subsection{Procedimento de Pesquisa}
\label{mtd-procedimento-pesquisa}

\section{Método do Estudo Exploratório}
\label{mtd-estudo-exploratorio}

% \begin{table}[h]
% 	\centering
% 	\caption[\textit{Strings} de busca utilizadas na revisão bibliográfica]{\textit{Strings} de busca utilizadas na revisão bibliográfica e a quantidade resultante de artigos analisados para cada busca.}
% 	\label{tabela-strings-busca}
% 	\begin{tabular}{lll}
% 		\hline
% 		\multicolumn{1}{c}{\textbf{Base de Dados}} & \multicolumn{1}{c}{\textit{\textbf{String de Busca}}}                                                                                                                                                    & \multicolumn{1}{c}{\textbf{\begin{tabular}[c]{@{}l@{}}Nº de Artigos\\ Analisados\end{tabular}}} \\ \hline
% 		Google Acadêmico   & \begin{tabular}[c]{@{}l@{}}data privacy\\ ``data engineering''\\ 2019-2024\end{tabular}    	& 20      \\ \hline
% 		Google Acadêmico   & \begin{tabular}[c]{@{}l@{}}data privacy\\ ``data engineering''\\ 2023-2024\end{tabular}    	& 10      \\ \hline
% 		Google Acadêmico   & \begin{tabular}[c]{@{}l@{}}software engineer data \\``data privacy'' 2019-2024\end{tabular}  & 30  \\ \hline
% 		Google Acadêmico   & \begin{tabular}[c]{@{}l@{}}``data privacy'' \\software engineer data \\OR data lineage \\OR access control \\OR metadata management \\2019-2024\end{tabular}         & 20           \\ \hline
% 		Google Acadêmico   & \begin{tabular}[c]{@{}l@{}}data privacy \\OR ``data engineering'' \\2019-2024\end{tabular} 																		  & 10    \\ \hline
% 		Google Acadêmico   & \begin{tabular}[c]{@{}l@{}}4th industrial revolution \\technology impact \\2019-2024\end{tabular} 	  & 20    \\ \hline
% 		Google Acadêmico   & \begin{tabular}[c]{@{}l@{}}self sovereign identity \\``data privacy''\\2019-2024\end{tabular} 	  & 10    \\ \hline
% 	\end{tabular}
% 	\legend{Fonte: Autor.}
% \end{table}

\section{Método da Modelagem de Privacidade}
\label{mtd-modelagem-privacidade}

% \begin{figure}[h]
% 	\centering
% 	\caption[Diagrama BPMN da Modelagem de um SIG]{Diagrama BPMN da Modelagem de um SIG para um atributo de qualidade.}
% 	\includegraphics[keepaspectratio=true,scale=0.30]{figuras/modelagem_sig.eps}
% 	\label{figura-modelagem-sig}
% 	\legend{Fonte: Autor.}
% \end{figure}

\section{Método Orientado a Provas de Conceito}
\label{mtd-provas-conceito}


\section{Método de Desenvolvimento}
\label{mtd-desenvolvimento}


\section{Método de Análise de Resultados}
\label{mtd-analise-resultados}


\section{Cronogramas de Atividades}
\label{mtd-cronograma}

Esta seção descreve o fluxo de atividades e subprocessos para a realização da primeira e da segunda etapas do trabalho, bem como os cronogramas de execução associados.

\subsection{Atividades e Subprocessos da Primeira Etapa}
\label{met-atividades-primeira-etapa}

% \begin{enumerate}
% 	\item \textbf{[Nome da Atividade 1]}: [Descrição da atividade];
% 	\begin{itemize}
% 		\item \textit{Status}: [Status da atividade], e
% 		\item \textit{Resultado}: [Resultado obtido].
% 	\end{itemize}
	
% 	\item \textbf{[Nome da Atividade 2]}: [Descrição da atividade];
% 	\begin{itemize}
% 		\item \textit{Status}: [Status da atividade], e
% 		\item \textit{Resultado}: [Resultado obtido].
% 	\end{itemize}
% 	% Adicione mais atividades conforme necessário
% \end{enumerate}

% A Tabela \ref{tabela-cronograma-primeira-etapa} apresenta o cronograma mensal de finalização das atividades. A Figura \ref{figura-primeira-etapa} ilustra o diagrama BPMN do fluxo de realização.

% \begin{table}[h]
% 	\centering
% 	\caption[Cronograma de execução da Primeira Etapa]{Cronograma de finalização de atividades.}
% 	\label{tabela-cronograma-primeira-etapa}
% 	\begin{tabular}{r|ccccc}
% 		\hline
% 		\multicolumn{1}{c}{\textbf{ATIVIDADE}} & \textbf{MÊS 1} & \textbf{MÊS 2} & \textbf{MÊS 3} & \textbf{MÊS 4} & \textbf{MÊS 5} \\ \hline
% 		[Atividade 1] & X & & & & \\ \hline
% 		[Atividade 2] & & X & & & \\ \hline
% 		% Adicione mais linhas conforme necessário
% 	\end{tabular}
% 	\legend{Fonte: Autor.}
% \end{table}

% \begin{figure}[h]
% 	\centering
% 	\caption[Diagrama BPMN da Primeira Etapa]{Diagrama BPMN do processo.}
% 	\includegraphics[keepaspectratio=true,scale=0.3]{figuras/diagrama_processo.eps}
% 	\label{figura-primeira-etapa}
% 	\legend{Fonte: Autor.}
% \end{figure}

\subsection{Atividades e Subprocessos da Segunda Etapa}
\label{met-atividades-segunda-etapa}

% \begin{enumerate}
% 	\item \textbf{[Nome da Atividade 1]}: [Descrição da atividade];
% 	\begin{itemize}
% 		\item \textit{Status}: [Status da atividade], e
% 		\item \textit{Resultado}: [Resultado obtido].
% 	\end{itemize}
% 	% Adicione mais atividades conforme necessário
% \end{enumerate}

% A Tabela \ref{tabela-cronograma-segunda-etapa} apresenta o cronograma mensal. A Figura \ref{figura-segunda-etapa} ilustra o diagrama BPMN do fluxo.

% \begin{table}[h]
% 	\centering
% 	\caption[Cronograma de execução da Segunda Etapa]{Cronograma para finalização de atividades.}
% 	\label{tabela-cronograma-segunda-etapa}
% 	\begin{tabular}{r|ccccc}
% 		\hline
% 		\multicolumn{1}{c}{\textbf{ATIVIDADE}} & \textbf{MÊS 1} & \textbf{MÊS 2} & \textbf{MÊS 3} & \textbf{MÊS 4} & \textbf{MÊS 5} \\ \hline
% 		[Atividade 1] & X & & & & \\ \hline
% 		[Atividade 2] & & X & & & \\ \hline
% 		% Adicione mais linhas conforme necessário
% 	\end{tabular}
% 	\legend{Fonte: Autor.}
% \end{table}

% \begin{figure}[h]
% 	\centering
% 	\caption[Diagrama BPMN da Segunda Etapa]{Diagrama BPMN do processo.}
% 	\includegraphics[keepaspectratio=true,scale=0.25]{figuras/diagrama_processo_2.eps}
% 	\label{figura-segunda-etapa}
% 	\legend{Fonte: Autor.}
% \end{figure}

\section{Resumo do Capítulo}
\label{mtd-resumo}

Lorem ipsum dolor sit amet, consectetur adipiscing elit. Sed scelerisque porta accumsan. Cras sed vestibulum lorem. Curabitur ac tincidunt urna. Praesent non faucibus risus. Praesent efficitur lacus et euismod venenatis. Nulla nec ante felis. Praesent ex libero, mollis sit amet commodo et, finibus id orci. Pellentesque ut mi accumsan sapien hendrerit maximus. Praesent iaculis sit amet erat a imperdiet. Maecenas ac leo pulvinar lacus posuere pharetra. Sed suscipit, arcu ut commodo elementum, ipsum tellus fermentum dolor, nec aliquam elit tellus et felis. Phasellus lacinia tempus tempus. In dictum mi purus, nec lobortis turpis consequat in. Etiam suscipit leo quis augue vulputate auctor.