\chapter{Metodologia}
\label{cap-metodologia}

% Este capítulo deve descrever detalhadamente os métodos e procedimentos utilizados na pesquisa
% Organize as seções de forma lógica, explicando cada etapa do trabalho

\section{Classificação da Pesquisa}
\label{mtd-classificacao-pesquisa}

% Utilize a figura abaixo para ilustrar os tipos de classificação da pesquisa
% \begin{figure}[h]
% \centering
% \caption[Classificação da Pesquisa Científica]{Tipos de classificação de pesquisa científica e as respectivas classificações deste trabalho.}
% \includegraphics[keepaspectratio=true,scale=0.17]{figuras/research_types.eps}
% \label{figura-tipos-pesquisa}
% \legend{Fonte: \cite{inmetro2003}. Figura elaborada pelo autor.}
% \end{figure}

\subsection{Abordagem de Pesquisa}
\label{mtd-abordagem-pesquisa}
% Descreva e justifique a abordagem escolhida:
% - Qualitativa
% - Quantitativa
% - Mista

\subsection{Objetivo de Pesquisa}
\label{mtd-objetivo-pesquisa}
% Classifique quanto aos objetivos:
% - Exploratória
% - Descritiva
% - Explicativa

\subsection{Natureza de Pesquisa}
\label{mtd-natureza-pesquisa}
% Indique e justifique a natureza:
% - Básica
% - Aplicada

\subsection{Procedimento de Pesquisa}
\label{mtd-procedimento-pesquisa}
% Descreva os procedimentos utilizados:
% - Pesquisa Bibliográfica
% - Estudo de Caso
% - Pesquisa Experimental
% - Outros

\section{Método do Estudo Exploratório}
\label{mtd-estudo-exploratorio}
% Descreva como foi realizada a revisão bibliográfica
% Utilize a tabela abaixo para documentar as strings de busca
% \begin{table}[h]
% 	\centering
% 	\caption[\textit{Strings} de busca utilizadas na revisão bibliográfica]{\textit{Strings} de busca utilizadas na revisão bibliográfica e a quantidade resultante de artigos analisados para cada busca.}
% 	\label{tabela-strings-busca}
% 	\begin{tabular}{lll}
% 		\hline
% 		\multicolumn{1}{c}{\textbf{Base de Dados}} & \multicolumn{1}{c}{\textit{\textbf{String de Busca}}}                                                                                                                                                    & \multicolumn{1}{c}{\textbf{\begin{tabular}[c]{@{}l@{}}Nº de Artigos\\ Analisados\end{tabular}}} \\ \hline
% 		Google Acadêmico   & \begin{tabular}[c]{@{}l@{}}data privacy\\ ``data engineering''\\ 2019-2024\end{tabular}    	& 20      \\ \hline
% 		Google Acadêmico   & \begin{tabular}[c]{@{}l@{}}data privacy\\ ``data engineering''\\ 2023-2024\end{tabular}    	& 10      \\ \hline
% 		Google Acadêmico   & \begin{tabular}[c]{@{}l@{}}software engineer data \\``data privacy'' 2019-2024\end{tabular}  & 30  \\ \hline
% 		Google Acadêmico   & \begin{tabular}[c]{@{}l@{}}``data privacy'' \\software engineer data \\OR data lineage \\OR access control \\OR metadata management \\2019-2024\end{tabular}         & 20           \\ \hline
% 		Google Acadêmico   & \begin{tabular}[c]{@{}l@{}}data privacy \\OR ``data engineering'' \\2019-2024\end{tabular} 																		  & 10    \\ \hline
% 		Google Acadêmico   & \begin{tabular}[c]{@{}l@{}}4th industrial revolution \\technology impact \\2019-2024\end{tabular} 	  & 20    \\ \hline
% 		Google Acadêmico   & \begin{tabular}[c]{@{}l@{}}self sovereign identity \\``data privacy''\\2019-2024\end{tabular} 	  & 10    \\ \hline
% 	\end{tabular}
% 	\legend{Fonte: Autor.}
% \end{table}

\section{Método da Modelagem}
\label{mtd-modelagem}

% \begin{figure}[h]
% 	\centering
% 	\caption[Diagrama BPMN da Modelagem de um SIG]{Diagrama BPMN da Modelagem de um SIG para um atributo de qualidade.}
% 	\includegraphics[keepaspectratio=true,scale=0.30]{figuras/modelagem_sig.eps}
% 	\label{figura-modelagem-sig}
% 	\legend{Fonte: Autor.}
% \end{figure}

\section{Método Orientado a Provas de Conceito}
\label{mtd-provas-conceito}


\section{Método de Desenvolvimento}
\label{mtd-desenvolvimento}
% Detalhe as etapas do desenvolvimento:
% - Ferramentas utilizadas
% - Processos seguidos
% - Técnicas aplicadas

\section{Método de Análise de Resultados}
\label{mtd-analise-resultados}
% Explique como os resultados serão:
% - Coletados
% - Analisados
% - Validados

\section{Cronogramas de Atividades}
\label{mtd-cronograma}

Esta seção descreve o fluxo de atividades e subprocessos para a realização da primeira e da segunda etapas do trabalho, bem como os cronogramas de execução associados.

\subsection{Atividades e Subprocessos da Primeira Etapa}
\label{met-atividades-primeira-etapa}
% Use o modelo de lista de atividades comentado como exemplo
% Inclua o cronograma e diagrama BPMN conforme modelos comentados
% \begin{enumerate}
% 	\item \textbf{[Nome da Atividade 1]}: [Descrição da atividade];
% 	\begin{itemize}
% 		\item \textit{Status}: [Status da atividade], e
% 		\item \textit{Resultado}: [Resultado obtido].
% 	\end{itemize}
	
% 	\item \textbf{[Nome da Atividade 2]}: [Descrição da atividade];
% 	\begin{itemize}
% 		\item \textit{Status}: [Status da atividade], e
% 		\item \textit{Resultado}: [Resultado obtido].
% 	\end{itemize}
% 	% Adicione mais atividades conforme necessário
% \end{enumerate}

% A Tabela \ref{tabela-cronograma-primeira-etapa} apresenta o cronograma mensal de finalização das atividades. A Figura \ref{figura-primeira-etapa} ilustra o diagrama BPMN do fluxo de realização.

% \begin{table}[h]
% 	\centering
% 	\caption[Cronograma de execução da Primeira Etapa]{Cronograma de finalização de atividades.}
% 	\label{tabela-cronograma-primeira-etapa}
% 	\begin{tabular}{r|ccccc}
% 		\hline
% 		\multicolumn{1}{c}{\textbf{ATIVIDADE}} & \textbf{MÊS 1} & \textbf{MÊS 2} & \textbf{MÊS 3} & \textbf{MÊS 4} & \textbf{MÊS 5} \\ \hline
% 		[Atividade 1] & X & & & & \\ \hline
% 		[Atividade 2] & & X & & & \\ \hline
% 		% Adicione mais linhas conforme necessário
% 	\end{tabular}
% 	\legend{Fonte: Autor.}
% \end{table}

% \begin{figure}[h]
% 	\centering
% 	\caption[Diagrama BPMN da Primeira Etapa]{Diagrama BPMN do processo.}
% 	\includegraphics[keepaspectratio=true,scale=0.3]{figuras/diagrama_processo.eps}
% 	\label{figura-primeira-etapa}
% 	\legend{Fonte: Autor.}
% \end{figure}

\subsection{Atividades e Subprocessos da Segunda Etapa}
\label{met-atividades-segunda-etapa}

% \begin{enumerate}
% 	\item \textbf{[Nome da Atividade 1]}: [Descrição da atividade];
% 	\begin{itemize}
% 		\item \textit{Status}: [Status da atividade], e
% 		\item \textit{Resultado}: [Resultado obtido].
% 	\end{itemize}
% 	% Adicione mais atividades conforme necessário
% \end{enumerate}

% A Tabela \ref{tabela-cronograma-segunda-etapa} apresenta o cronograma mensal. A Figura \ref{figura-segunda-etapa} ilustra o diagrama BPMN do fluxo.

% \begin{table}[h]
% 	\centering
% 	\caption[Cronograma de execução da Segunda Etapa]{Cronograma para finalização de atividades.}
% 	\label{tabela-cronograma-segunda-etapa}
% 	\begin{tabular}{r|ccccc}
% 		\hline
% 		\multicolumn{1}{c}{\textbf{ATIVIDADE}} & \textbf{MÊS 1} & \textbf{MÊS 2} & \textbf{MÊS 3} & \textbf{MÊS 4} & \textbf{MÊS 5} \\ \hline
% 		[Atividade 1] & X & & & & \\ \hline
% 		[Atividade 2] & & X & & & \\ \hline
% 		% Adicione mais linhas conforme necessário
% 	\end{tabular}
% 	\legend{Fonte: Autor.}
% \end{table}

% \begin{figure}[h]
% 	\centering
% 	\caption[Diagrama BPMN da Segunda Etapa]{Diagrama BPMN do processo.}
% 	\includegraphics[keepaspectratio=true,scale=0.25]{figuras/diagrama_processo_2.eps}
% 	\label{figura-segunda-etapa}
% 	\legend{Fonte: Autor.}
% \end{figure}

\section{Resumo do Capítulo}
\label{mtd-resumo}
% Sintetize os principais aspectos da metodologia
% Reforce a adequação dos métodos aos objetivos do trabalho