\begin{resumo}
%  O resumo deve ressaltar o objetivo, o método, os resultados e as conclusões 
%  do documento. A ordem e a extensão
%  destes itens dependem do tipo de resumo (informativo ou indicativo) e do
%  tratamento que cada item recebe no documento original. O resumo deve ser
%  precedido da referência do documento, com exceção do resumo inserido no
%  próprio documento. (\ldots) As palavras-chave devem figurar logo abaixo do
%  resumo, antecedidas da expressão Palavras-chave:, separadas entre si por
%  ponto e finalizadas também por ponto. O texto pode conter no mínimo 150 e 
%  no máximo 500 palavras, é aconselhável que sejam utilizadas 200 palavras. 
%  E não se separa o texto do resumo em parágrafos.

Lorem ipsum dolor sit amet, consectetur adipiscing elit. Sed scelerisque porta accumsan. Cras sed vestibulum lorem. Curabitur ac tincidunt urna. Praesent non faucibus risus. Praesent efficitur lacus et euismod venenatis. Nulla nec ante felis. Praesent ex libero, mollis sit amet commodo et, finibus id orci. Pellentesque ut mi accumsan sapien hendrerit maximus. Praesent iaculis sit amet erat a imperdiet. Maecenas ac leo pulvinar lacus posuere pharetra. Sed suscipit, arcu ut commodo elementum, ipsum tellus fermentum dolor, nec aliquam elit tellus et felis. Phasellus lacinia tempus tempus. In dictum mi purus, nec lobortis turpis consequat in. Etiam suscipit leo quis augue vulputate auctor.

 \vspace{\onelineskip}
    
 \noindent
 \textbf{Palavras-chave}: A. B. C.
\end{resumo}
