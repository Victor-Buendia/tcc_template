\begin{errata}
% A errata é um elemento opcional que lista correções de erros presentes no trabalho
% Deve ser inserida logo após a folha de rosto, quando necessária
% Se não houver errata, deixe este arquivo em branco

Elemento opcional da \citeonline[4.2.1.2]{NBR14724:2011}. \textbf{Caso não 
deseje uma errata, deixar todo este arquivo em branco}. Exemplo:

\vspace{\onelineskip}

% Primeiro, cite a referência completa do trabalho
FERRIGNO, C. R. A. \textbf{Tratamento de neoplasias ósseas apendiculares com
reimplantação de enxerto ósseo autólogo autoclavado associado ao plasma
rico em plaquetas}: estudo crítico na cirurgia de preservação de membro em
cães. 2011. 128 f. Tese (Livre-Docência) - Faculdade de Medicina Veterinária e
Zootecnia, Universidade de São Paulo, São Paulo, 2011.

% Em seguida, apresente a tabela com as correções
\begin{table}[htb]
\center
\footnotesize
\begin{tabular}{|p{1.4cm}|p{1cm}|p{3cm}|p{3cm}|}
  \hline
   \textbf{Folha} & \textbf{Linha}  & \textbf{Onde se lê}  & \textbf{Leia-se}  \\
    \hline
    1 & 10 & auto-conclavo & autoconclavo\\
   \hline
\end{tabular}
\end{table}

\end{errata}
